\documentclass[french, a4paper, 12pt, titlepage]{article}
%% Peut remplacer "article" par "scrartcl" %%

\usepackage{a4wide}
%\usepackage[top=2cm, bottom=2cm, left=2cm, right=2cm]{geometry}
\raggedbottom%  Prevents vertical white space on pages that cannot be filled properly

\usepackage{hyperref}
\hypersetup{%
	colorlinks=true,       	% false: boxed links; true: colored links
	linkcolor=black,          	% color of internal links
	urlcolor=blue,           	% color of external links
	citecolor=grey
}

\usepackage[T1]{fontenc}
%\usepackage{fourier}
%\usepackage{utopia}
%\usepackage{palatino}

\usepackage{lmodern}
%% ajouter fonte petite capitale grasse à lmodern avec celle de computer modern %%
\rmfamily
\DeclareFontShape{T1}{lmr}{b}{sc}{<->ssub*cmr/bx/sc}{}
\DeclareFontShape{T1}{lmr}{bx}{sc}{<->ssub*cmr/bx/sc}{}
%% /ajout %%
\usepackage{wrapfig}

%\usepackage[a4paper]{geometry} % marges plus petites que a4paper standard
\usepackage{listings} % insérer code source
\lstloadlanguages{sh,bash,awk,make}
%\usepackage{algorithm} % algorithmique
%\usepackage{algorithmic}
\usepackage{url}
\usepackage[usenames, dvipsnames]{color} % couleurs (nombre de base étendu)
\usepackage{graphicx} % insérer images
\usepackage[utf8]{inputenc}
\usepackage[french]{babel}
\usepackage{amsmath}
\usepackage{amsfonts}
\usepackage{amssymb}
\usepackage{amsthm}
\usepackage{multicol}
\definecolor{grey}{rgb}{0.96,0.96,0.96}
\definecolor{grey2}{rgb}{0.3,0.3,0.3}

%% Define listings params %%
\lstset{%
%	numbers=left,
	language=bash,
	tabsize=4,
	frame=single, % cadre autour du code
	breaklines=true, % autorise couper ligne trop longue
	basicstyle=\small\ttfamily,
	numberstyle=\scriptsize\ttfamily,
	backgroundcolor=\color{grey},
	showstringspaces=false,
	keywordstyle=\color{OliveGreen},
	stringstyle=\color{BrickRed},
	commentstyle=\color{grey2}\it,
	stepnumber=1 % numérote toute les x lignes
}
% listing utf8 fr %
\lstset{%
	inputencoding=utf8,
	extendedchars=true,
	literate=
		{é}{{\'{e}}}1
		{è}{{\`{e}}}1
		{ê}{{\^{e}}}1
		{ë}{{\"{e}}}1
		{û}{{\^{u}}}1
		{ù}{{\`{u}}}1
		{â}{{\^{a}}}1
		{à}{{\`{a}}}1
		{î}{{\^{\i}}}1
		{ç}{{\c{c}}}1
		{Ç}{{\c{C}}}1
		{É}{{\'{E}}}1
		{Ê}{{\^{E}}}1
		{À}{{\`{A}}}1
		{Â}{{\^{A}}}1
		{Î}{{\^{I}}}1
}
%% /Define listings params %%

%% Francisation des algorithmes
%\renewcommand{\algorithmicrequire} {\textbf{\textsc{Entrées:}}}
%\renewcommand{\algorithmicensure}  {\textbf{\textsc{Sorties:}}}
%\renewcommand{\algorithmicwhile}   {\textbf{tant que}}
%\renewcommand{\algorithmicdo}      {\textbf{faire}}
%\renewcommand{\algorithmicendwhile}{\textbf{fin tant que}}
%\renewcommand{\algorithmicend}     {\textbf{fin}}
%\renewcommand{\algorithmicif}      {\textbf{si}}
%\renewcommand{\algorithmicendif}   {\textbf{fin si}}
%\renewcommand{\algorithmicelse}    {\textbf{sinon}}
%\renewcommand{\algorithmicthen}    {\textbf{alors}}
%\renewcommand{\algorithmicfor}     {\textbf{pour}}
%\renewcommand{\algorithmicforall}  {\textbf{pour tout}}
%\renewcommand{\algorithmicdo}      {\textbf{faire}}
%\renewcommand{\algorithmicendfor}  {\textbf{fin pour}}
%\renewcommand{\algorithmicloop}    {\textbf{boucler}}
%\renewcommand{\algorithmicendloop} {\textbf{fin boucle}}
%\renewcommand{\algorithmicrepeat}  {\textbf{répéter}}
%\renewcommand{\algorithmicuntil}   {\textbf{jusqu'à}}
%\renewcommand{\algorithmiccomment} {\STATE //}
%\newcommand{\BEGIN}{\STATE \fbox{Début}}
%\newcommand{\END}{\STATE \fbox{Fin}}
%\floatname{algorithm}{Algorithme}
%% /francisation des algorithmes

\renewcommand{\qedsymbol}{}

\newcommand{\petit}[1]{%
	\medskip \noindent
	\begin{small}
	#1)
	\end{small}
}

\begin{document}
\title{Introduction à Git}
%\author{\includegraphics{clubnix}}
\author{Club*Nix}
\date{\url{https://clubnix.fr}}

\maketitle
%% Laisse page blanche pour verso page de garde %%

\vfill
\pagebreak

%\tableofcontents
\newpage
\strut\thispagestyle{empty}
\vfill
\pagebreak
\tableofcontents
\strut\thispagestyle{empty}
%\setcounter{page}{0}
\newpage
\setcounter{page}{1}

\part{Introduction}

\section{Les gestionnaires de versions}

\paragraph{} Un gestionnaire de version est un outil permettant de maintenir plusieurs versions, c'est-à-dire qu'il est capable d'enregistrer les différents états d'un ou de plusieurs fichiers au cours du temps et de naviguer facilement entres ces états.

\paragraph{}Il en existe plusieurs, les plus connus étant \emph{mercurial}, \emph{SVN} et \emph{Git}.  Nous nous intéresserons uniquement à ce dernier qui est notamment utilisé pour la réalisation de Linux.

\section{Git, un gestionnaire décentralisé}

\paragraph{} Git à la particularité, contrairement à d'autres gestionnaires de version d'être décentralisé, c'est à dire qu'il n'a pas besoin d'un serveur pour fonctionner.  Cependant pour des raisons de simplicité il est souvent utilisé de cette manière.
%TODO ref plus loin

\paragraph{} Chaque personne possède un dépôt privé, et un dépôt public.  Le dépôt privé sert à faire du développement local, et le public sert à envoyer des versions du code source afin que d'autres personnes puissent les télécharger.  En donnant à tout un groupe le même dépôt public, nous obtenons un système centralisé.


\part{Installation}
\paragraph{}L'installation de Git differe en fonction de votre système d'exploitation:
\section{Linux}
\paragraph{} Pour installer Git sur un système Linux il suffi d'executer la commande suivante:
\begin{lstlisting}
$ sudo apt-get install git-core gitk
\end{lstlisting}
\paragraph{}A savoir que Git est installé par defaut sur certain système Linux.

\section{Windows}
\paragraph{}Pour les systèmes Windaub il faut installer \emph{msysgit} qui installera un terminal Unix permettant d'utiliser les commandes Linux de Git (et donc de suivre la suite du tuto) ainsi que les paquet Git essentiel a son bon fonctionnement, pour installer msysgit reportez vous au tutoriel fourni sur \url{http://msysgit.github.io}

\section{MacOS}
\paragraph{}Il y a plusieurs façons d’installer Git sous Mac OS X. Le plus simple est de se baser sur cet installeur pour Mac OS X : \url{https://code.google.com/archive/p/git-osx-installer/}. MacOS utilisant un terminal Linux nous pourront utiliser les commandes données dans ce tuto sous MacOS.

\part{Configuration de Git}
\paragraph{}Avant de commencer a utiliser Git il vous faudra creer une configuration basique de Git, pour cela il vous faudra seulement executer quelque commande dans un terminal Linux:
\begin{lstlisting}
$ git config --global color.diff auto
$ git config --global color.status auto
$ git config --global color.branch auto
\end{lstlisting}
\paragraph{}Ces commandes vont mettre la configuration par defaut, nous allons maintenant configurer plus en detail \emph{votre} Git en ajoutant votre nom et votre mail:
\begin{lstlisting}
$ git config --global user.name "VotrePseudo"
$ git config --global user.email moi@email.com
\end{lstlisting}


\part{Création de dépôt local}

\section{Clonage}

\paragraph{} La manière la plus fréquente de créer un dépôt
(\emph{repository}) Git est le ``clonage''~: la copie entière d'un projet géré avec Git. Si l'on prends par exemple le projet de ce cours même qui est stocké sur \href{https://github.com}{GitHub} à l'adresse
\url{https://github.com/ClubNix/git}, il suffit de taper la commande suivante pour copier le projet entier dans le dossier courant~:

\begin{lstlisting}
$ git clone https://github.com/ClubNix/git
\end{lstlisting}
\paragraph{}Il aussi possible de manière facultative de spécifier dans quel dépôt le clonage va se faire~:
\begin{lstlisting}
$ git clone https://github.com/ClubNix/git dossier
\end{lstlisting}

\section{Création d'un dépôt vide}

\paragraph{} Il est bien sûr possible avec Git de créer un dépôt vide. Pour cela, il suffit de faire la commande suivante dans le dossier creer pour stocker le depot Git~:

\begin{lstlisting}
$ git init
\end{lstlisting}

\paragraph{} Cela va tout simplement ajouter les fichiers propres à Git dans le répertoire courant et ainsi d'initialiser la gestion de version.

\part{Exemple d'un workflow avec Git}
\paragraph{} Avec Git, un workflow courant, est constitué de 4 étapes : la modification, la validation, le commit et l'envoi. Nous allons voir chacune des étapes dans cette partie.
\section{Modification}
\paragraph{}La modification est tout simplement le moment pendant lequel les fichiers (de code, etc\dots) sont modifiés. C'est la partie qui semble la plus simple dans la gestion de version. Cependant, afin d'exploiter la gestion de version à son plein potentiel, certaines règles doivent être respectées.
\paragraph{} Premièrement, Git analyse les différents changement en faisant des ``diff'', qui consistent à regarder le différences entre un même fichier à des versions différentes. Ainsi, afin de clarifier un peu le tout, il est conseillé de préférer du texte sur plusieurs ligne que du texte sur une seule ligne.
\paragraph{} Ensuite, afin de pouvoir mieux gérer son dépôt mais aussi de mieux gérer la stabilité et la fonctionnalité du projet, il est aussi conseillé de maximiser le nombre de version et d'en valider une aussitôt que le projet ``fonctionne''. Cela facilitera grandement la détection et la résolution d'erreurs mais aussi le travail avec les branches que l'on verra plus tard.

\part{Les commandes Git de base}
\paragraph{}Arrivé a ce stade vous savez creer un nouveau depot ou en cloner un existant afin de travailler dessus, nous allons donc maintenant voir les commandes de base pour effectuer des "sauvegardes" de votre dossier Git:

\section{Modifier le code et effectuer des commits}
\subsection{git status}
\paragraph{}Lorsque vous modifiez les fichiers contenus dans votre depot Git vous devez les commit sur le serveur afin de les sauvegarder pour l'ensemble des utilisateur du depot, pour cela vous pouvez utiliser la commande suivante afin de voir les fichiers qui ont été modifié mais dont les modifications ne sont pas enregistrer sur le dépot:
\begin{lstlisting}
$ git status
\end{lstlisting}

\paragraph{}Pour aller plus loin nous pouvons même voir en détail les modification non enregistrée grace a la commande suivante:
\begin{lstlisting}
$ git diff
\end{lstlisting}

\subsection{git commit}
\paragraph{}Maintenant que vous savez quelle modification sont a enregistrer nous pouvons passer au commit de ces changements. Pour cela plusieurs methodes sont possible:

\paragraph{}La premiere va consister a ajouter les fichier qui ont modifier dans la file des fichiers a commit:
\begin{lstlisting}
$ git add nomfichier1 nomfichier2
\end{lstlisting}
\paragraph{}Puis de commit ces fichiers grace a la commande:
\begin{lstlisting}
$ git commit
\end{lstlisting}
\paragraph{}NB: Apres un \emph{git add ...} si vous faite un \emph{git status} vous verrez les fichiers ajoutés apparaitre en vert.

\paragraph{}La deuxieme methodes consiste a « commiter » tous les fichiers qui étaient listés dans \emph{git status} dans les colonnes « Changes to be committed » et « Changed but not updated » (qu’ils soient en vert ou en rouge) grace a la commande :
\begin{lstlisting}
$ git commit -a
\end{lstlisting}

\paragraph{}La troisieme solution va consister a indiquer a Git quel fichiers doivent etre commit :
\begin{lstlisting}
$ git commit nomFichier1 nomFichier2
\end{lstlisting}

\subsection{Annuler un commit}
\paragraph{}Dans le cas ou vous auriez commiter des erreurs il est encore possible d'annuler le dernier commit effectué.
\paragraph{}Pour cela nous allons commencer par vérifier les logs:
\begin{lstlisting}
$ git log
\end{lstlisting}
\paragraph{}Vous devriez normalement voir tous les derniers commit effectués. Chaque commit est numéroté grâce à un long numéro hexadécimal comme 12328a1bcbf231da8eaf942f8d68c7dc0c7c4f38. Cela permet de les identifier.
\paragraph{}Pour aller plus loin vous pouvez voir les lignes modifiées par le commit en utilisant la commande suivante :
\begin{lstlisting}
$ git log -p
\end{lstlisting}
\paragraph{}Vous pouvez aussi avoir un résumé plus court des commits avec :
\begin{lstlisting}
$ git log --stat
\end{lstlisting}

\paragraph{}Dans le cas ou vous auriez fait une faute lors de votre dernier commit il existe une commande permettant d'ouvrir un editeur de texte pour modifié le dernier commit effectue:
\begin{lstlisting}
$ git commit --amend
\end{lstlisting}
\paragraph{}Cette commande est généralement utilisée juste après avoir effectué un commit lorsqu’on se rend compte d’une erreur dans le message. Il est en effet impossible de modifier le message d’un commit lorsque celui-ci a été transmis à d’autres personnes.
\paragraph{}Pour annuler votre dernier commit vous pouvez utiliser la commande suivante:
\begin{lstlisting}
$ git reset HEAD^
\end{lstlisting}
\paragraph{}Cela annule le dernier commit et revient à l’avant-dernier.
\paragraph{}Pour indiquer à quel commit on souhaite revenir, il existe plusieurs notations :
\begin{lstlisting}
-HEAD : dernier commit ;
-HEAD^ : avant-dernier commit ;
-HEAD^^ : avant-avant-dernier commit ;
-HEAD~2 : avant-avant-dernier commit (notation équivalente) ;
-d6d98923868578a7f38dea79833b56d0326fcba1 : indique un numéro de commit précis.
\end{lstlisting}
\paragraph{}Seul le commit est retiré de Git ; vos fichiers, eux, restent modifiés. Vous pouvez alors à nouveau changer vos fichiers si besoin est et refaire un commit.
\paragraph{}Vous pouvez aussi annuler un commit et \emph{toutes} les modifications qu'il implique grace a la commande:
\begin{lstlisting}
$ git reset --hard HEAD^
\end{lstlisting}

\subsection{Restaurer un fichier a son etat anterieur avant le commit}
\paragraph{}Si vous avez modifié plusieurs fichiers mais que vous n’avez pas encore envoyé le commit et que vous voulez restaurer un fichier tel qu’il était au dernier commit, utilisez:
\begin{lstlisting}
$ git checkout nomfichier
\end{lstlisting}
\paragraph{}Il est possible de retirer un fichier qui avait été ajouté pour être « commité » en procédant comme suit:
\begin{lstlisting}
$ git reset HEAD -- fichier_a_supprimer
\end{lstlisting}

\section{Partager votre travail et télécharger les nouveautés}
\paragraph{}Sous Git vous travaillez sur un depot local, mais tous les fichiers du projet sont aussi contenus sur un depot distant dans lequel tous les contribueurs peuvent faire des modifications. Il sera donc necessaire de savoir télécharger les ouveauté apportées par d'autre mais aussi de savoir envoyer vos nouveautés sur le depot distant.
\subsection{Télécharger les nouveautés}
\paragraph{}Pour télécharger les nouveautés du depot il vous suffit d'utiliser la commande suivante:
\begin{lstlisting}
$ git pull
\end{lstlisting}
\paragraph{}Deux cas sont possibles :
\begin{itemize}
\item Soit vous n’avez effectué aucune modification depuis le dernier pull, dans ce cas la mise à jour est simple (on parle de mise à jour fast-forward) ;
\item Soit vous avez fait des commits en même temps que d’autres personnes. Les changements qu’ils ont effectués sont alors fusionnés aux vôtres automatiquement.
\end{itemize}

Si deux personnes modifient en même temps deux endroits distincts d’un même fichier, les changements sont intelligemment fusionnés par Git.

Parfois, mais cela arrive normalement rarement, deux personnes modifient la même zone de code en même temps. Dans ce cas, Git dit qu’il y a un conflit car il ne peut décider quelle modification doit être conservée ; il vous indique alors le nom des fichiers en conflit. Ouvrez-les avec un éditeur et recherchez une ligne contenant « <<<<<<<<< ». Ces symboles délimitent vos changements et ceux des autres personnes. Supprimez ces symboles et gardez uniquement les changements nécessaires, puis faites un nouveau commit pour enregistrer tout cela.

\subsection{Envoyer vos commit}
\paragraph{}Avant d'envoyer quoi que ce soit nous vous conseillons fortement de vérifier ce que vous allez envoyer grace a un \emph{git log -p}, mais aussi de télécharger ce qui se trouve sur le dépot distant car le serveur ne peut régler les conflits à votre place s’il y en a. Personne ne doit avoir fait un push avant vous depuis votre dernier pull.

Une fois cela fait vous pouvez utiliser la commande suivante pour publier vos commit:
\begin{lstlisting}
$ git push
\end{lstlisting}

\subsection{Annuler un commit publié}
\paragraph{}Si par megarde vous avez publié des erreurs vous pouvez toujours annuler le dernier commit publié, pour cela vous devez connaitre le numero hexadécimal de votre commit grace a \emph{git log}, ensuite vous pouvez utiliser la commande suivante pour annuler votre commit:
\begin{lstlisting}
$ git revert numeroHexadecimal
\end{lstlisting}

\part{Travailler avec des branches}
\paragraph{}Les branches font partie du cœur même de Git et constituent un de ses principaux atouts. C’est un moyen de travailler en parallèle sur d’autres fonctionnalités. C’est comme si vous aviez quelque part une « copie » du code source du site qui vous permet de tester vos idées les plus folles et de vérifier si elles fonctionnent avant de les intégrer au véritable code source de votre projet.

Dans Git, toutes les modifications que vous faites au fil du temps sont par défaut considérées comme appartenant à la branche principale appelée « master » :


%TODO biblio
%
% http://git-scm.com <- référence
% http://rogerdudler.github.io/git-guide/
% http://stackoverflow.com/questions/315911/git-for-beginners-the-definitive-practical-guide#320140
% https://www.atlassian.com/git/
%
%TODO
% Explain why `git init --bare' if on own server
\end{document}
% vim: spell : spelllang=fr