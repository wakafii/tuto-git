\documentclass[french, a4paper, 12pt, titlepage]{article}
%% Peut remplacer "article" par "scrartcl" %%

\usepackage{a4wide}
%\usepackage[top=2cm, bottom=2cm, left=2cm, right=2cm]{geometry}
\raggedbottom % prevents vertical white space on pages that cannot be filled properly

\usepackage{hyperref}
\hypersetup{
	colorlinks=true,       	% false: boxed links; true: colored links
	linkcolor=black,          	% color of internal links
	urlcolor=blue,           	% color of external links
	citecolor=grey
}

\usepackage[T1]{fontenc}
%\usepackage{fourier}
%\usepackage{utopia}
%\usepackage{palatino}

\usepackage{lmodern}
%% ajouter fonte petite capitale grasse à lmodern avec celle de computer modern %%
\rmfamily
\DeclareFontShape{T1}{lmr}{b}{sc}{<->ssub*cmr/bx/sc}{}
\DeclareFontShape{T1}{lmr}{bx}{sc}{<->ssub*cmr/bx/sc}{}
%% /ajout %%
\usepackage{wrapfig}

%\usepackage[a4paper]{geometry} % marges plus petites que a4paper standard
\usepackage{listings} % insérer code source
\lstloadlanguages{sh,bash,awk,make}
%\usepackage{algorithm} % algorithmique
%\usepackage{algorithmic}
\usepackage{url}
\usepackage[usenames, dvipsnames]{color} % couleurs (nombre de base étendu)
\usepackage{graphicx} % insérer images
\usepackage[utf8]{inputenc}
\usepackage[french]{babel}
\usepackage{amsmath}
\usepackage{amsfonts}
\usepackage{amssymb}
\usepackage{amsthm}
\usepackage{multicol}
\definecolor{grey}{rgb}{0.96,0.96,0.96}
\definecolor{grey2}{rgb}{0.3,0.3,0.3}

%% Define listings params %%
\lstset{
%	numbers=left,
	language=bash,
	tabsize=4,
	frame=single, % cadre autour du code
	breaklines=true, % autorise couper ligne trop longue
	basicstyle=\small\ttfamily,
	numberstyle=\scriptsize\ttfamily,
	backgroundcolor=\color{grey},
	showstringspaces=false,
	keywordstyle=\color{OliveGreen},
	stringstyle=\color{BrickRed},
	commentstyle=\color{grey2}\it,
	stepnumber=1 % numérote toute les x lignes
}
% listing utf8 fr %
\lstset{%
	inputencoding=utf8,
	extendedchars=true,
	literate=
		{é}{{\'{e}}}1
		{è}{{\`{e}}}1
		{ê}{{\^{e}}}1
		{ë}{{\¨{e}}}1
		{û}{{\^{u}}}1
		{ù}{{\`{u}}}1
		{â}{{\^{a}}}1
		{à}{{\`{a}}}1
		{î}{{\^{i}}}1
		{ç}{{\c{c}}}1
		{Ç}{{\c{C}}}1
		{É}{{\'{E}}}1
		{Ê}{{\^{E}}}1
		{À}{{\`{A}}}1
		{Â}{{\^{A}}}1
		{Î}{{\^{I}}}1
}
%% /Define listings params %%

%% Francisation des algorithmes
%\renewcommand{\algorithmicrequire} {\textbf{\textsc{Entrées:}}}
%\renewcommand{\algorithmicensure}  {\textbf{\textsc{Sorties:}}}
%\renewcommand{\algorithmicwhile}   {\textbf{tant que}}
%\renewcommand{\algorithmicdo}      {\textbf{faire}}
%\renewcommand{\algorithmicendwhile}{\textbf{fin tant que}}
%\renewcommand{\algorithmicend}     {\textbf{fin}}
%\renewcommand{\algorithmicif}      {\textbf{si}}
%\renewcommand{\algorithmicendif}   {\textbf{fin si}}
%\renewcommand{\algorithmicelse}    {\textbf{sinon}}
%\renewcommand{\algorithmicthen}    {\textbf{alors}}
%\renewcommand{\algorithmicfor}     {\textbf{pour}}
%\renewcommand{\algorithmicforall}  {\textbf{pour tout}}
%\renewcommand{\algorithmicdo}      {\textbf{faire}}
%\renewcommand{\algorithmicendfor}  {\textbf{fin pour}}
%\renewcommand{\algorithmicloop}    {\textbf{boucler}}
%\renewcommand{\algorithmicendloop} {\textbf{fin boucle}}
%\renewcommand{\algorithmicrepeat}  {\textbf{répéter}}
%\renewcommand{\algorithmicuntil}   {\textbf{jusqu'à}}
%\renewcommand{\algorithmiccomment} {\STATE //}
%\newcommand{\BEGIN}{\STATE \fbox{Début}}
%\newcommand{\END}{\STATE \fbox{Fin}}
%\floatname{algorithm}{Algorithme}
%% /francisation des algorithmes

\renewcommand{\qedsymbol}{}

\newcommand{\petit}[1]{
	\medskip \noindent
	\begin{small}
	#1)
	\end{small}
}

\begin{document}
\title{Introduction à Git}
\author{\includegraphics{clubnix}}
\date{\url{https://clubnix.fr}}

\maketitle
%% Laisse page blanche pour verso page de garde %%

\vfill
\pagebreak

%\tableofcontents
\newpage
\strut\thispagestyle{empty}
\vfill
\pagebreak
\tableofcontents
\strut\thispagestyle{empty}
%\setcounter{page}{0}
\newpage
\setcounter{page}{1}

\part{Introduction}

\section{Les gestionnaires de versions}

\paragraph{} Un gestionnaire de version est un outil permettant de maintenir
plusieurs versions, c'est-à-dire qu'il est capable d'enregistrer les différents
états d'un ou de plusieurs fichiers au cours du temps et de naviguer facilement
entres ces états.

\paragraph{} Il en existe plusieurs, les plus connus étant \emph{mercurial},
\emph{SVN} et \emph{Git}.  Nous nous intéresserons uniquement à ce dernier qui
est notamment utilisé pour la réalisation de Linux.

\section{Git, un gestionnaire décentralisé}

\paragraph{} Git à la particularité, contrairement à d'autres gestionnaires de
version d'être décentralisé, c'est à dire qu'il n'a pas besoin d'un serveur
pour fonctionner.  Cependant pour des raisons de simplicité il est souvent
utilisé de cette manière.
%TODO ref plus loin

\paragraph{} Chaque personne possède un dépôt privé, et un dépôt public.  Le
dépôt privé sert à faire du développement local, et le public sert à envoyer
des versions du code source afin que d'autres personnes puissent les
télécharger.  En donnant à tout un groupe le même dépôt public, nous obtenons
un système centralisé.

\part{Création de dépôt local}

\section{Clonage}

\paragraph{} La manière la plus fréquente de créer un dépôt
(\emph{repository}) Git est le ``clonage''~: la copie entière d'un projet géré
avec Git. Si l'on prends par exemple le projet de ce cours même qui est stocké
sur \href{https://github.com}{GitHub} à l'adresse
\url{https://github.com/ClubNix/git}, il suffit de taper la commande suivante
pour copier le projet entier dans le dossier courant~:

\begin{lstlisting}
$ git clone https://github.com/ClubNix/git
\end{lstlisting}

\paragraph{} Il aussi possible de manière facultative de spécifier dans quel
dépôt le clonage va se faire~:

\begin{lstlisting}
$ git clone https://github.com/ClubNix/git dossier
\end{lstlisting}

\section{Création d'un dépôt vide}

\paragraph{} Il est bien sûr possible avec Git de créer un dépôt vide. Pour
cela, il suffit de faire~:

\begin{lstlisting}
$ git init
\end{lstlisting}

\paragraph{} Cela va tout simplement ajouter les fichiers propres à Git dans le
répertoire courant et ainsi d'initialiser la gestion de version.

\part{Exemple d'un workflow avec Git}

\paragraph{} Avec Git, un workflow courant, est constitué de 4 étapes : la
modification, la validation, le commit et l'envoi. Nous allons voir chacune des
étapes dans cette partie.

\section{Modification}

\paragraph{} La modification est tout simplement le moment pendant lequel les
fichiers (de code, etc\dots) sont modifiés. C'est la partie qui semble la plus
simple dans la gestion de version. Cependant, afin d'exploiter la gestion de
version à son plein potentiel, certaines règles doivent être respectées.

\paragraph{} Premièrement, Git analyse les différents changement en faisant des
``diff'', qui consistent à regarder le différences entre un même fichier à des
versions différentes. Ainsi, afin de clarifier un peu le tout, il est conseillé
de préférer du texte dur plusieurs ligne que du texte sur une seule ligne.

\paragraph{} Ensuite, afin de pouvoir mieux gérer son dépôt mais aussi de mieux
gérer la stabilité et la fonctionnalité du projet, il est aussi conseillé de
maximiser le nombre de version et d'en valider une aussitôt que le projet
``fonctionne''. Cela facilitera grandement la détection et la résolution
d'erreurs mais aussi le travail avec le branche que l'on verra plus tard.

%TODO biblio
%
% http://git-scm.com <- réference
% http://rogerdudler.github.io/git-guide/
% http://stackoverflow.com/questions/315911/git-for-beginners-the-definitive-practical-guide#320140
%
%TODO
% Explain why `git init --bare' if on own server
\end{document}
% vim: spell : spelllang=fr
